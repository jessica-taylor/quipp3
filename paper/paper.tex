\documentclass{article}
\usepackage{amsmath}
\usepackage{amsfonts}
\usepackage{multicol} \usepackage{fancyheadings} \usepackage{pdfpages}
\usepackage{nips15submit_e,times}
\setlength{\emergencystretch}{10em}

\newtheorem{theorem}{Theorem}[section]
\newtheorem{lemma}[theorem]{Lemma}
\newtheorem{proposition}[theorem]{Proposition}
\newtheorem{corollary}[theorem]{Corollary}
 
\newenvironment{proof}[1][Proof]{\begin{trivlist}
\item[\hskip \labelsep {\bfseries #1}]}{\end{trivlist}}
\newenvironment{definition}[1][Definition]{\begin{trivlist}
\item[\hskip \labelsep {\bfseries #1}]}{\end{trivlist}}
\newenvironment{example}[1][Example]{\begin{trivlist}
\item[\hskip \labelsep {\bfseries #1}]}{\end{trivlist}}
\newenvironment{remark}[1][Remark]{\begin{trivlist}
\item[\hskip \labelsep {\bfseries #1}]}{\end{trivlist}}

\DeclareMathOperator*{\Exp}{\mathbb{E}}
\DeclareMathOperator*{\Prob}{\mathbf{P}}


\title{Qualitative Probabilistic Programming}


\author{
Jessica Taylor \\
Department of Computer Science\\
Stanford University \\
Stanford, CA 94305
\texttt{jessica.liu.taylor@gmail.com}
\And
Andreas Stuhlm\"uller \\
Department of Brain and Cognitive Sciences \\
MIT \\
Cambridge, MA 02139 \\
\texttt{andreas@stuhlmueller.org}
\And
Noah Goodman \\
Department of Psychology \\
Stanford University \\
Stanford, CA 94305
\texttt{ngoodman@stanford.edu}
}

\newcommand{\fix}{\marginpar{FIX}}
\newcommand{\new}{\marginpar{NEW}}

%\nipsfinalcopy % Uncomment for camera-ready version

\begin{document}

  \maketitle

  \begin{abstract}
    In probabilistic programs, sometimes it is difficult to specify the correct
    parameterized family of distributions.  We explore an extension to
    probabilistic programming languages that allows programmers to mark some
    distributions as unspecified.  Then, we can fill in the distribution with
    some family and infer parameters.
  \end{abstract}

  \section{Introduction}

  By separating model specification and inference, probabilistic programming 
  has made it easier for non-experts to implement and use probabilistic models.
  Practitioners frequently have strong intuitions about the {\em structure}
  of their domain knowledge, such as which latent variables exist and what
  their causal relations are, and probabilistic programming allows them to encode
  this knowledge. However, it also requires them to specify the specific parametric
  shape  and parameterization of any distributions used, and intuitions tend to
  be much less precise there.
  We present Quipp, a system that does {\em not} require such specification;
  instead, random variables and random functions can be left undefined
  and will automatically be filled in under maximum entropy assumptions
  based on their types and available datasets.
  
  Our formalism can concisely express a wide variety of models that machine
  learning practitioners care about, and we provide an expectation
  maximization algorithm that can learn the parameters for many of
  these models with reasonable efficiency. This system makes it easy
  for non-experts to encode their beliefs about the data and to get
  predictions based on as few additional assumptions as possible.
  
  % This feature has multiple advantages.  First, it is easier to write
  % models without knowing about the class of models being used.  This should
  % make probabliistic programming more accessible to non-experts.  Secondly,
  % parameter inference is more efficient if specialized algorithms are used
  % rather than the generic algorithms used to infer other random variables
  % (such as Metropolis Hastings).

  % We define an example probabilistic programming language with this feature
  % (Quipp) and show how it can be used to write machine learning models
  % concisely.

  In an ordinary probabilistic programming language (such as Church),
  it is possible to treat parameters as random variables.  This
  would allow ordinary inference algorithms to infer parameters.  However,
  there are advantages of having unknown functions as a feature
  in the language.
  First, it is easier to
  write programs without knowing the details of different parameterized distributions.
  Second, the system can use specialized algorithms to infer parameters faster.

  % Inference in these models is performed using the expectation-maximization
  % algorithm, with alternating steps of inferring latent variables and optimizing
  % parameters.

  % - Motivation for Quipp
  %   - Explanation of "unknown functions"
  %   - Writing machine learning algorithms as probabilistic programs
  %   - Accessibility to non-experts
  %   - Comparison to existing probabilistic programming languages
  %     - In other languages, use random variables for parameters
  %     - Random variables slower because they are updated independently

  In the following, we first specify the syntax used to write Quipp programs,
  including the notation for unknown variables and functions.
  We describe the class of exponential family variables and functions that our system can learn,
  and present the expectation maximization algorithm used to learn them.
  We then demonstrate the expressiveness of our language, and the broad
  applicability of our algorithm, by writing some of the most common machine learning models
  in Quipp: clustering, naive Bayes, factor analysis, a Hidden Markov model, Latent Dirichlet Allocation, and
  a neural net.
  
  \section{Syntax}

  Quipp is implemented as a library for webppl programs.  Webppl [TODO cite] is a probabilistic programming language
  implemented in Javascript. Quipp programs are written as webppl programs that have access to special functions.

  Here is an example of a Quipp program to cluster 2d points into 3 clusters:


  \begin{verbatim}
var nclusters = 3;
var dim = 2;

var n = 100;

testParamInference(function(randFunction) {
  var pointType = Vector(dim, Double);
  var getPoint = randFunction(Categorical(nclusters), pointType);
  return function() {
    return repeat(n, function(i) {
      var cluster = randomInteger(nclusters);
      return observe('point' + i, getPoint, cluster);
    });
  };
});
  \end{verbatim}

  It is written as a generative model producing the observations.  Notice that we declared two types (\texttt{PointType} and \texttt{ClusterType})
  and one random function \texttt{getPoint}).
  Type annotations are necessary for random functions.  The type
  \texttt{Vector(2, Double)} expresses the fact that the points are 2-dimensional,
  and the type \texttt{Categorical(3)} expresses the fact that there are 3
  possible clusters (so a cluster may be either 0, 1, or 2).  We assume that
  the distribution over clusters is uniform, but we do not know the distribution
  of points in each cluster.
  We will fill in the
  \texttt{getPoint} function with a learned function that will take a random
  sample from the given cluster.

  \section{Family of distributions}
  
  For unknown functions, the family of random functions used is a kind
    of generalized linear model.  We assume that the distribution
    of the function's return value
    is some exponential family whose natural
    parameters are determined from the arguments:

    $$p_{\eta}(y | x) = \exp\left(\eta(x)^T \phi(y) - g(\eta(x))\right)$$

    Here, $\eta(x)$ is the natural parameter, $\phi(y)$ is a vector of $y$'s sufficient statistics,
    and $g$ is the log partition function.

    To determine $\eta(x)$, we label
    some subset of the sufficient statistics of both $x$ and $y$ as \emph{features}.  The natural
    parameters corresponding to non-features are constant, while natural
    parameters corresponding to features are determined as an affine
    function of the features of the arguments.  Features are the same as sufficient
    statistics for the categorical distribution, but while both $X$ and $X^2$ are
    sufficient statistics for the Gaussian distribution, only $X$ is a feature.

    Let $\psi(x)$ be the features of $x$.  Then
    $$p_{\mathbf{N}}(y | x) = \exp\left(\begin{bmatrix} 1 \\ \psi(x) \end{bmatrix} ^T \mathbf{N} \phi(y) - g\left(\mathbf{N}^T \begin{bmatrix} 1 \\ \psi(x) \end{bmatrix}\right)\right)$$

    where $\mathbf{N}$ is a matrix containing our parameters.  It must have 0 for each entry whose row corresponds
    to a sufficient statistics of $y$ that is not a feature and whose column is not 1.
    This ensures that only the
    natural parameters that are features of $y$ are affected by $x$.


  \section{Inference}

    To infer both latent variables and parameters, we run the
    expectation maximization algorithm on the probabilistic model, iterating stages of
    estimating latent variables using Metropolis Hastings and inferring
    parameters using gradient descent.

    For the expectation step, we must estimate latent variable distributions given
    fixed values for the parameters.  To do this, we can run the Quipp program
    with random functions set to use these fixed parameter values to generate their results.
    We use the Metropolis Hastings algorithm to perform inference in this program,
    yielding traces.  Next, for each random function, we can find all calls
    to it in the trace to get the training data.

    For the maximization step, given samples from each random function, we set the
    parameters of the function to maximize the likelihood of the samples.  To do this we,
    we use gradient descent.

    Given $(x, y)$ samples, parameter estimation to maximize log probability is a convex
    problem because the log probability function is concave:
    $$\log p_{\mathbf{N}}(y | x) = \begin{bmatrix} 1 \\ \psi(x) \end{bmatrix} ^T \mathbf{N} \phi(y) - g\left(\mathbf{N}^T \begin{bmatrix} 1 \\ \psi(x) \end{bmatrix}\right)$$

    This relies on the fact that $g$ is convex, but this is true in general for any exponential family distribution.
    Since the problem is convex, it is possible to use gradient descent to optimize the parameters.  Although
    the only exponential family distributions we use in this paper are the categorical and Gaussian distributions,
    we can use the same algorithms for other exponential families, such as the Poisson and gamma distributions.

  \section{Evaluation}

    To evaluate performance, for each model, we:
    \begin{itemize}
      \item
        Randomly generate parameters $\theta$
      \item
        Generate datasets $x_{train}, x_{test}$ using $\theta$
      \item
        Estimate $\log P(x_{test} | \theta)$
      \item
        Use the EM algorithm to infer approximate parameters $\hat{\theta}$ from $x_{train}$
      \item
        Estimate $\log P(x_{test} | \hat{\theta})$ and compare to $\log P(x_{test} | \theta)$
    \end{itemize}
    Estimating $\log P(x_{test} | \theta)$ is nontrivial, given that the model contains latent variables.
    We use the Sequential Monte Carlo algorithm for this.  Between observations,


  \section{Examples}

  \subsection{Clustering}
{\small
\begin{verbatim}
var nclusters = 3;
var dim = 2;

var n = 100;

testParamInference(function(randFunction) {
  var pointType = Vector(dim, Double);
  var getPoint = randFunction(Categorical(nclusters), pointType);
  return function() {
    return repeat(n, function(i) {
      var cluster = randomInteger(nclusters);
      return observe('point' + i, getPoint, cluster);
    });
  };
});
\end{verbatim}
}

In this example, we cluster 2d points into 3 different clusters.  Given a cluster, the distribution for a point is some independent Gaussian distribution.  This is similar to fuzzy c-means clustering.

We randomly generated parameters for this example 100 times, and each time took 100 samples and then ran 10 EM iterations.  The accuracy is defined as the maximum percentage of points assigned to the correct cluster, for any permutation of clusters.  On average, accuracy increased in each EM iteration, as shown it this graph:

\begin{center}
\includegraphics[scale=0.4]{cluster_accuracy.png}
\end{center}

\subsection{Naive Bayes}

{\small
\begin{verbatim}
nfeatures = 10
FeaturesType = Vector(nfeatures, Bool)

get_class = rand_function(Unit, Bool)

class_1_features = rand_function(Unit, FeaturesType)
class_2_features = rand_function(Unit, FeaturesType)

def sample():
  which_class = get_class()
  if which_class:
    features = class_1_features()
  else:
    features = class_2_features()
  return (which_class, features)
\end{verbatim}
}

The naive Bayes model is similar to the clustering model.  We have two classes and a feature
distribution for each.  Since each feature is boolean, we will learn
a different categorical distribution for each class.

(figure should show average classification accuracy)

  \subsection{Factor analysis}
{\small
\begin{verbatim}
num_components = 3
point_dim = 5

ComponentsType = Vector(num_components, Double)
PointType = Vector(point_dim, Double)

get_point = rand_function(ComponentsType, PointType)

def sample():
  components = [normal(0, 1)
                for i in range(num_components)]
  return (comenents, get_point(components))
\end{verbatim}
}

The factor analysis model is very similar to the clustering model.  The main difference is that we replace the categorical \texttt{ClusterType} type with a vector type.  This results in the model attempting to find each point as an affine function of a vector of standard normal values.

(figure should show accuracy by EM iteration.  Accuracy can be measured by distances between different factor coefficients, when ordered according to some permutation)
  \subsection{Hidden Markov model}
{\small
\begin{verbatim}
num_states = 5
num_observations = 3
sequence_length = 10

StateType = Categorical(num_states)
ObsType = Categorical(num_observations)

trans_fun = rand_function(StateType, StateType)
obs_fun = rand_function(StateType, ObsType)

def sample():
  states = [uniform(num_states)]
  for i in range(sequence_length - 1):
    states.append(trans_fun(states[-1]))
  return (states, [obs_fun(s) for s in states])
\end{verbatim}
}

In this example, we use the unknown function \texttt{trans\_fun} for state transitions and \texttt{obs\_fun} for observations.  This means that we will learn both the state transitions and the observation distribution.


\subsection{Latent Dirichlet allocation}
{\small
\begin{verbatim}
n_classes = 10
n_words = 100
n_words_per_document = 1000

ClassType = Categorical(n_classes)
WordType = Categorical(n_words)

class_to_word = rand_function(ClassType, WordType)

def sample():
    which_class = uniform_categorical(n_classes)
    words = [class_to_word(which_class)
             for i in range(n_words_per_document)]
    return (which_class, words)
\end{verbatim}
}

In this example, we use the unknown function \texttt{class\_to\_word} to map classes to word distributions.  Note that each column of the matrix of parameters for \texttt{class\_to\_word} will represent a categorical distribution over words, and there will be one column for each class.

(figure should show accuracy over time.  Accuracy can be measured as distance between the learned categorical distributions, for some permutation of classes)

\subsection{Neural network}

{\small
\begin{verbatim}
input_dim = 100
hidden_dim = 20
output_dim = 2

InputType = Categorical(input_dim)
HiddenType = Categorical(hidden_dim)
OutputType = Categorical(output_dim)

input_to_hidden = rand_function(InputType, HiddenType)
hidden_to_output = rand_function(HiddenType, OutputType)

def sample(input_layer):
  hidden_layer = input_to_hidden(input_layer)
  output_layer = hidden_to_output(hidden_layer)
  return ((), output_layer)
\end{verbatim}
}

\subsection{A more complex model}

TODO: if there is time, we should put a more complex data science type example, where we add dependencies and show change in accuracy as we add/remove assumptions.


  \section{Discussion}

  We have found that it is possible to write many useful machine learning models as Quipp programs and then use generic algorithms for inference.  Furthermore, performance is <???>.  This should make it much easier for non-experts to write useful machine learning models.
  
  In the future, it will be useful to expand the set of types supported.  It is possible to define reasonable default distributions for non-recursive algebraic data types, and it may also be possible to define them for recursive algebraic data types using catamorphisms.  Also, it will be useful to create a more usable interface to infer parameters and perform additional data pracessing given these parameters.


\end{document}

